\chapter{Исследовательский раздел}

\section{Описание исследования}

Целью исследования является определение зависимости времени получения и сохранения изображений различных размеров от типа СУБД (на примере PostgreSQL и MongoDB).
Для каждого изображения время получения и вставки замеряется 100 раз, после чего берется среднее значение.

\section{Результаты исследования}

Результаты исследования времени сохранения изображений различных размеров в зависимости от типа СУБД представлены в таблице~\ref{table:measure2}.
\begin{table}[!ht]
	\centering
	\caption{\label{table:measure2} Количественные данные, полученные в результате исследования}
	\begin{tabularx}{\textwidth}{|X|X|X|}
		\hline
		Размер изображения (Кбайт) & Время сохранения изображения в PostgreSQL (мкс) & Время получения изображения в MongoDB (мкс) \\ \hline
      	19612 & 522108 & 49090 \\ \hline
		16208 & 368289 & 32572 \\ \hline
		11047 & 251669 & 22274 \\ \hline
		7494 & 173262 & 16101 \\ \hline
		5101 & 113389 & 13625 \\ \hline
		3482 & 67236 & 20543 \\ \hline
		2388 & 46250 & 15806 \\ \hline
		1647 & 34640 & 10678 \\ \hline
		1060 & 20348 & 7880 \\ \hline
		859 & 16409 & 7549 \\ \hline
		477 & 9295 & 5662 \\ \hline
		325 & 6693 & 4208 \\ \hline
		224 & 5060 & 3729 \\ \hline
		156 & 4417 & 3372 \\ \hline
	\end{tabularx}
\end{table}

\clearpage
 
На рисунке~\ref{img:measure2.pdf} представлен график зависимости времени сохранения изображений различных размеров от типа СУБД.
\includeimage
	{measure2.pdf}
	{f}
	{h}
	{1\textwidth}
	{Зависимость времени сохранения изображений различных размеров от типа СУБД}
	
\clearpage

Результаты исследования времени получения изображений различных размеров в зависимости от типа СУБД представлены в таблице~\ref{table:measure1}.
\begin{table}[!ht]
	\centering
	\caption{\label{table:measure1} Количественные данные, полученные в результате исследования}
	\begin{tabularx}{\textwidth}{|X|X|X|}
		\hline
		Размер изображения (Кбайт) & Время получения изображения в PostgreSQL (мкс) & Время получения изображения в MongoDB (мкс) \\ \hline
      	19612 & 10911 & 195 \\ \hline
		16208 & 7914 & 155 \\ \hline
		11047 & 6240 & 119 \\ \hline
		7494 & 5316 & 79 \\ \hline
		5101 & 3718 & 70 \\ \hline
		3482 & 3408 & 64 \\ \hline
		2388 & 1971 & 57 \\ \hline
		1647 & 1466 & 53 \\ \hline
		1060 & 614 & 48 \\ \hline
		859 & 590 & 45 \\ \hline
		477 & 279 & 45 \\ \hline
		325 & 182 & 45 \\ \hline
		224 & 139 & 44 \\ \hline
		156 & 111 & 44 \\ \hline
	\end{tabularx}
\end{table}

\clearpage

На рисунке~\ref{img:measure1.pdf} представлен график зависимости времени получения изображений различных размеров от типа СУБД.
\includeimage
	{measure1.pdf}
	{f}
	{h}
	{1\textwidth}
	{Зависимость времени получения изображений различных размеров от типа СУБД}

\section*{Вывод}

В исследовательском разделе было проведено исследование зависимости времени получения и сохранения изображения от типа СУБД (на примере PostgreSQL и MongoDB).

Исходя из полученных в таблицах~\ref{table:measure2},~\ref{table:measure1} результатов, был сделан вывод, что время сохранения изображений, размер которых превышает 5000 Кбайт, в СУБД MongoDB в $\sim$11 раз меньше, чем в СУБД PostgreSQL, которая относится к типу реляционных баз данных, и в $\sim$1.5--3 раза меньше для меньших размерностей.

Аналогично время получения изображений, размер которых больше 3000 КБайт, в СУБД MongoDB в $\sim$50 раз меньше, чем в СУБД PostgreSQL, и в $\sim$5--17 раз меньше для меньших размерностей.


