\chapter{Конструкторский раздел}

В конструкторском разделе будут спроектированы диаграмма вариантов использования, диаграмма базы данных и ограничения целостности.

\section{Диаграмма вариантов использования}

На рисунке \ref{img:usecase} представлена спроектированная диаграмма вариантов использования.
\includeimage
	{usecase}
	{f}
	{H}
	{1\textwidth}
	{Диаграмма вариантов использования}
	
Исходя из спроектированной диаграммы вариантов использования~\ref{img:usecase}, выделяются следующие роли:
\begin{enumerate}
	\item Гость~-- неавторизованный сотрудник компании;
	\item Сотрудник компании~-- пользователь, зарегистрированный в приложении;
	\item Сотрудник службы безопасности (СБ)~-- сотрудник компании, обеспечивающий безопасность в рамках компании.
\end{enumerate}
	
\section{Диаграмма базы данных}

На рисунке \ref{img:entity-rel} представлена спроектированная диаграмма базы данных и ограничения целостности.
\includeimage
	{entity-rel}
	{f}
	{H}
	{0.41\textwidth}
	{Диаграмма базы данных и ограничения целостности}

Для проектируемой базы данных необходимы следующие таблицы:
\begin{enumerate}
	\item таблица компаний~-- <<\texttt{company}>>;
	\item таблица сотрудников компании~-- <<\texttt{employee}>>;
	\item таблица информационных карточек о сотруднике~-- <<\texttt{info\_card}>>;
	\item таблица документов, удостоверяющих личность~-- <<\texttt{document}>>;
	\item таблица полей документов, удостоверяющих личность~-- <<\texttt{field}>>;
	\item таблица фотографий сотрудников из документов, удостоверяющих личность~-- <<\texttt{photo}>>;
	\item таблица контрольно-пропускных пунктов (КПП)~-- <<\texttt{checkpoint}>>;
	\item связующая таблица проходов сотрудников через КПП~-- <<\texttt{passage}>>.
\end{enumerate}

\section{Вывод}

В конструкторском разделе были спроектированы диаграмма вариантов использования, диаграмма базы данных и ограничения целостности.