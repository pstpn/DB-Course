\chapter{Конструкторский раздел}
	
\section{Диаграмма базы данных}



Для проектируемой базы данных необходимы следующие таблицы:
\begin{enumerate}
	\item таблица компаний~-- <<\texttt{company}>>;
	\item таблица сотрудников компании~-- <<\texttt{employee}>>;
	\item таблица информационных карточек о сотрудниках~-- <<\texttt{info\_card}>>;
	\item таблица документов, удостоверяющих личность~-- <<\texttt{document}>>;
	\item таблица полей документов, удостоверяющих личность~-- <<\texttt{field}>>;
	\item таблица фотографий сотрудников из документов, удостоверяющих личность~-- <<\texttt{photo}>>;
	\item таблица контрольно-пропускных пунктов (КПП)~-- <<\texttt{checkpoint}>>;
	\item связующая таблица проходов сотрудников через КПП~-- <<\texttt{passage}>>.
\end{enumerate}

В таблице \ref{table:companyCols} описаны отношения таблицы компаний <<\texttt{company}>>.
\begin{table}[h!]
	\begin{center}
		\caption{\label{table:companyCols} Отношения таблицы <<\texttt{company}>>}
		\begin{tabularx}{\textwidth}{|X|X|X|}
			\hline
			Атрибут & Тип & Описание \\ \hline
			\texttt{id} & \texttt{integer} & Идентификатор компании \\ \hline
			\texttt{name} & \texttt{text} & Название компании \\ \hline
			\texttt{city} & \texttt{text} & Город расположения офиса/предприятия компании \\ \hline
		\end{tabularx}
	\end{center}
\end{table}

В таблице \ref{table:employeeCols} описаны отношения таблицы сотрудников <<\texttt{employee}>>.
\begin{table}[h!]
	\begin{center}
	\caption{\label{table:employeeCols} Отношения таблицы <<\texttt{employee}>>}
		\begin{tabularx}{\textwidth}{|X|X|X|}
			\hline
			Атрибут & Тип & Описание \\ \hline
			\texttt{id} & \texttt{integer} & Идентификатор сотрудника \\ \hline
			\texttt{phone\_number} & \texttt{text} & Номер телефона сотрудника \\ \hline
			\texttt{full\_name} & \texttt{text} & Полное имя сотрудника \\ \hline
			\texttt{company\_id} & \texttt{integer} & Идентификатор компании, в которой сотрудник работает \\ \hline
			\texttt{post} & \texttt{text} & Должность сотрудника\\ \hline
			\texttt{password} & \texttt{text} & Пароль сотрудника \\ \hline
			\texttt{refresh\_token} & \texttt{text} & Токен для обновления токена доступа к сайту \\ \hline
			\texttt{token\_expired\_at} & \texttt{text} & Время деактивации refresh\_token \\ \hline
			\texttt{date\_of\_birth} & \texttt{date} & Дата рождения сотрудника \\ \hline
		\end{tabularx}
	\end{center}
\end{table}

\clearpage

В таблице \ref{table:infoCardCols} описаны отношения таблицы информационных карточек о сотрудниках <<\texttt{info\_card}>>.
\begin{table}[h!]
	\begin{center}
	\caption{\label{table:infoCardCols} Отношения таблицы <<\texttt{info\_card}>>}
		\begin{tabularx}{\textwidth}{|X|X|X|}
			\hline
			Атрибут & Тип & Описание \\ \hline
			\texttt{id} & \texttt{integer} & Идентификатор карточки \\ \hline
			\texttt{created\_employee\_id} & \texttt{integer} & Идентификатор сотрудника, создавшего карточку \\ \hline
			\texttt{is\_confirmed} & \texttt{boolean} & Подтверждена ли карточка сотрудником СБ \\ \hline
			\texttt{created\_date} & \texttt{date} & Дата создания карточки \\ \hline
		\end{tabularx}
	\end{center}
\end{table}

В таблице \ref{table:documentCols} описаны отношения таблицы документов, удостоверяющих личность <<\texttt{document}>>.
\begin{table}[h!]
	\begin{center}
		\caption{\label{table:documentCols} Отношения таблицы <<\texttt{document}>>}
		\begin{tabularx}{\textwidth}{|X|X|X|}
			\hline
			Атрибут & Тип & Описание \\ \hline
			\texttt{id} & \texttt{integer} & Идентификатор документа \\ \hline
			\texttt{serial\_number} & \texttt{text} & Серийный номер документа \\ \hline
			\texttt{info\_card\_id} & \texttt{integer} & Идентификатор информационной карточки, к которой привязан документ \\ \hline
			\texttt{type} & \texttt{text} & Тип документа \\ \hline
		\end{tabularx}
	\end{center}
\end{table}

\clearpage

В таблице \ref{table:fieldCols} описаны отношения таблицы полей документов, удостоверяющих личность <<\texttt{field}>>.
\begin{table}[h!]
	\begin{center}
		\caption{\label{table:fieldCols} Отношения таблицы <<\texttt{field}>>}
		\begin{tabularx}{\textwidth}{|X|X|X|}
			\hline
			Атрибут & Тип & Описание \\ \hline
			\texttt{id} & \texttt{integer} & Идентификатор поля документа \\ \hline
			\texttt{document\_id} & \texttt{integer} & Идентификатор документа, которому принадлежит поле \\ \hline
			\texttt{type} & \texttt{text} & Тип поля \\ \hline
			\texttt{value} & \texttt{text} & Значение поля \\ \hline
		\end{tabularx}
	\end{center}
\end{table}

В таблице \ref{table:photoCols} описаны отношения таблицы фотографий сотрудников из документов, удостоверяющих личность <<\texttt{photo}>>.
\begin{table}[h!]
	\begin{center}
		\caption{\label{table:photoCols} Отношения таблицы <<\texttt{photo}>>}
		\begin{tabularx}{\textwidth}{|X|X|X|}
			\hline
			Атрибут & Тип & Описание \\ \hline
			\texttt{id} & \texttt{integer} & Идентификатор фотографии \\ \hline
			\texttt{document\_id} & \texttt{integer} & Идентификатор документа, которому принадлежит фотография \\ \hline
			\texttt{key} & \texttt{text} & Ключ для получения доступа к фотографии \\ \hline
		\end{tabularx}
	\end{center}
\end{table}

\clearpage

В таблице \ref{table:passageCols} описаны отношения связующей таблицы проходов сотрудников через КПП <<\texttt{passage}>>.
\begin{table}[h!]
	\begin{center}
		\caption{\label{table:passageCols} Отношения таблицы <<\texttt{passage}>>}
		\begin{tabularx}{\textwidth}{|X|X|X|}
			\hline
			Атрибут & Тип & Описание \\ \hline
			\texttt{id} & \texttt{integer} & Идентификатор прохода \\ \hline
			\texttt{checkpoint\_id} & \texttt{integer} & Идентификатор КПП, через который был осуществлен проход \\ \hline
			\texttt{document\_id} & \texttt{integer} & Идентификатор документа, по которому был осуществлен проход \\ \hline
			\texttt{type} & \texttt{text} & Тип прохода (вход или выход) \\ \hline
			\texttt{time} & \texttt{timestamp} & Время прохода \\ \hline
		\end{tabularx}
	\end{center}
\end{table}

В таблице \ref{table:checkpointCols} описаны отношения таблицы КПП <<\texttt{checkpoint}>>.
\begin{table}[h!]
	\begin{center}
		\caption{\label{table:checkpointCols} Отношения таблицы <<\texttt{checkpoint}>>}
		\begin{tabularx}{\textwidth}{|X|X|X|}
			\hline
			Атрибут & Тип & Описание \\ \hline
			\texttt{id} & \texttt{integer} & Идентификатор КПП \\ \hline
			\texttt{phone\_number} & \texttt{text} & Номер стационарного телефона КПП \\ \hline
		\end{tabularx}
	\end{center}
\end{table}

\section{Ролевая модель}

Исходя из спроектированной диаграммы вариантов использования~\ref{img:usecase}, выделяются следующие роли:
\begin{enumerate}
	\item гость (неавторизованный сотрудник компании);
	\item сотрудник компании(пользователь, зарегистрированный в приложении);
	\item сотрудник службы безопасности (сотрудник компании, обеспечивающий безопасность в рамках компании/СБ);
	\item администратор.
\end{enumerate}

В таблице~\ref{table:roles} описаны режимы доступа к данным для представленных выше ролей.
\begin{table}[!ht]
	\centering
	\caption{\label{table:roles} Режимы доступа к данным для ролей базы данных}
	\begin{tabularx}{\textwidth}{|X|X|X|}
		\hline
		Роль & Таблицы с доступом на просмотр &  Таблицы с доступом на изменение \\ \hline
		\texttt{гость} & - & - \\ \hline
		\texttt{сотрудник компании} & \texttt{employee}, \texttt{info\_card}, \texttt{document}, \texttt{field}, \texttt{photo} & \texttt{employee}, \texttt{info\_card}, \texttt{document}, \texttt{field}, \texttt{photo} \\ \hline
		\texttt{сотрудник СБ} & все таблицы & \texttt{info\_card}, \texttt{passage} \\ \hline
		\texttt{админ} & все таблицы & все таблицы \\ \hline
	\end{tabularx}
\end{table}

\section{Триггер базы данных}

В течение дня важно отслеживать проходы сотрудников через КПП в целях безопасности и контроля рабочего дня~\cite{introCPP}.

Для предоставления нужной информации необходимо реализовать триггер базы данных, который после очередного прохода через КПП будет выводить информацию об общем количестве входов и выходов на территории предприятия/офиса.

Схема алгоритма работы триггера представлена на рисунке~\ref{img:trigger}.
\includeimage
	{trigger}
	{f}
	{H}
	{0.85\textwidth}
	{Схема алгоритма работы триггера базы данных}

\section*{Вывод}

В конструкторском разделе были спроектированы диаграмма базы данных, ограничения целостности, ролевая модель и триггер базы данных.