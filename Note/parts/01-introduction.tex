\chapter*{ВВЕДЕНИЕ}
\addcontentsline{toc}{chapter}{ВВЕДЕНИЕ}

В настоящее время все больше компаний вводят систему контрольно-пропускного режима, что объясняется некоторыми основными причинами~\cite{introCPP}:
\begin{enumerate}
	\item контроль времени прихода и ухода работника;
	\item обеспечение только санкционированного доступа в офис (здание, помещение);
	\item обеспечение сохранности имущества работодателя.
\end{enumerate}

Целью курсовой работы является разработка базы данных для идентификации пользователей при осуществлении контрольно-пропускного режима с использованием документов, удостоверяющих личность.

Для достижения поставленной цели необходимо решить следующие задачи:
\begin{enumerate}
	\item формализовать задачу и данные;
	\item провести анализ существующих СУБД и методов хранения изображений на основе
	формализованной задачи, спроектировать диаграмму вариантов использования;
	\item спроектировать диаграмму базы данных, ограничения целостности и ролевую модель;
	\item выбрать средства реализации базы данных и приложения (в том числе выбор СУБД и средств для хранения изображений);
	\item разработать базу данных и приложение;
	\item описать интерфейс доступа к базе данных;
	\item провести исследование зависимости времени получения и сохранения изображения от типа СУБД.
\end{enumerate}